%
% vwa.tex
% Beispieldokument/Vorlage für die vwa.cls von Alexander Leithner
% Dieses Dokument fällt nicht unter die LPPL, unter der die vwa.cls steht!
%

% Haupteinstellungen für den Textsatz der VWA:
% a4paper             gibt an, dass die Arbeit auf A4 (Norm EN ISO 216) gesetzt wird
% 12pt                gibt an, dass die Schriftgröße 12 Punkt betragen soll
% oneside             gibt an, dass die Arbeit einseitig und nicht im Duplex-Verfahren gesetzt werden soll
% BCOR=               gibt an, dass die Bindungskorrektur dem Wert hinter "=" entsprechen soll (immer mit Einheit anzugeben)
% parskip=            gibt den Abstand zwischen zwei Absätzen an (mögl. Werte: off, half, full, never)
% bibliography=totoc  gibt an, dass das Literaturverzeichnis im Inhaltsverzeichnis aufscheinen soll.
\documentclass[a4paper,12pt,oneside,BCOR=0mm,bibliography=totoc,parskip=half]{vwa}

% Lädt die Silbentrennungseinstellungen für die neue deutsche Rechtschreibung
% Um weitere Silbentrennungen zu laden, die Sprachen VOR deutsch eintragen (mit Komma getrennt)
% Bei anderssprachigen Arbeiten die jeweilige Sprache als letzte eintragen (mit Komma getrennt)
\usepackage[naustrian]{babel}

% Erzeugt österreichische Anführungszeichen („ und “)
\usepackage[style=austrian]{csquotes}


% --- FORMELSATZ ----------------------------------------------------------------------
% Wird der Formelsatz nicht benötigt, kann alles hier bis zu "allgemeine Information"
% entweder kommentiert (mittels Voranstellen eines % in jeder Zeile) oder gelöscht
% werden.
% Lädt zusätzliche Werkzeuge für den Formelsatz.
\usepackage{mathtools}

% Lädt die American Mathematical Society-Erweiterungen bzgl. Formelsatz:
% amsmath   sind generelle LaTeX-Erweiterungen für den fortgeschrittenen Formelsatz.
% amsfonts  sind moderne Schriften für Formeln incl. skalierbaren Zeichen.
% amssymb   sind erweiterte Zeichensätze für den fortgeschrittenen Formelsatz.
\usepackage{amsmath}
\usepackage{amsfonts}
\usepackage{amssymb}

% --- ALLGEMEINE INFORMATION ----------------------------------------------------------
%  Titel, Untertitel und VerfasserIn:
%% Ist kein Untertitel vorhanden, sind die geschwungenen Klammern für diesen Befehl
%% unbedingt leer zu lassen und nicht zu löschen!
\title{}
\subtitle{}
\author{}
\klasse{}
%  BetreuerIn und Schule:
%% Das Schullogo ist auf jeden Fall mit Dateiendung (am besten ".png") anzugeben!
%% Der Befehl "\schule" akzeptiert folgende Eingaben: [logo]{name}{adresse}
\betreuerLabel{Betreuer}
\betreuer{}
\schule[]{}{}
%  Ort der Schule, Abgabedatum:
%% Lt. Ministerium hat das Abgabedatum folgendes Format zu haben: "Monat Jahr"
%% Der Ort der Schule sollte mit der Postleitzahlangabe in der Adresse übereinstimmen.
\ort{}
\date{}

% --- BESONDERE TYPOGRAPHISCHE EINSTELLUNGEN ------------------------------------------
%  Zeilenabstand:
%% Der Zeilenabstand muss als dekadische Zahl angegeben werden, bspw. 1.5 für eineinhalb-
%% zeiligen Zeilenabstand.
\zeilenabstand{1.5}
%  Kopf- und Fußzeilen:
%% Folgende Kommandos können durch Entfernen des Prozentzeichens gültig gemacht werden
%% und haben so folgende Auswirkungen:
%%   \keinTitelImKopf   entfernt den Titel der Arbeit aus der Kopfzeile
%%   \titelImFuss       setzt Autor und Titel in die Fußzeile
%%   \schuleImFuss      setzt Autor und Schule in die Fußzeile
%%   \titelImKopfKlein  verkleinert den Titel der Arbeit in der Kopfzeile auf kleinst-
%%                      mögliche, noch lesbare Schriftgröße
%% Beachten Sie, dass \schuleImFuss nur funktioniert, wenn \titelImFuss nicht aufgerufen
%% wird, und dass \titelImKopfKlein nicht funktioniert, wenn \keinTitelImKopf gesetzt
%% wurde. Untenstehend können die einzelnen Kommandos auskommentiert werden:
% \keinTitelImKopf
% \titelImKopfKlein
% \titelImFuss
% \schuleImFuss

% --- BIBLIOGRAPHIE-DATENBANKEN -------------------------------------------------------
% Bibliographie-Datenbanken (.bib Dateien) können hier dem Dokument angefügt werden. Es
% empfiehlt sich, eine generelle Datei für die gesamte Arbeit anzulegen, oder jede aus
% den diversen Bibliothekssystemen (bspw. Nationalbibliothek, Universitätsbibliothek Wien)
% mittels der Funktion "BibTeX exportieren" heruntergeladene Dateien hier anzufügen.
\addbibresource{vwa.bib}

% --- DOKUMENT ------------------------------------------------------------------------
% Hier beginnt das tatsächliche Dokument. Hier sollte kein Inhalt eingefügt werden,
% sondern die einzelnen Kapitel stattdessen mittels "\input" eingebunden werden, wie
% mit dem Abstract und dem Vorwort gezeigt.
\begin{document}
\maketitle

\frontmatter
\chapter*{Abstract}\label{chapter:Abstract}
\addcontentsline{toc}{chapter}{Abstract}
%
% _abstract.tex
% Beispieldatei mitgeliefert mit der vwa.cls von Alexander Leithner
% Dieses Dokument fällt nicht unter die LPPL, unter der die vwa.cls steht!
% 

% Datei, in der das Abstract der VWA formuliert wird. Es empfiehlt sich, für jedes
% andere Kapitel eine neue Datei anzulegen, um Umstellungen in der VWA (im Haupt-
% dokument "vwa.tex" im Hauptordner des Projekts) zu erleichtern.

Hier wird das Abstract formuliert. Dieser Text sollte dabei gelöscht werden.


\chapter*{Vorwort}\label{chapter:Vorwort}
\addcontentsline{toc}{chapter}{Vorwort}
%
% _vorwort.tex
% Beispieldatei mitgeliefert mit der vwa.cls von Alexander Leithner
% Dieses Dokument fällt nicht unter die LPPL, unter der die vwa.cls steht!
% 

% Datei, in der das Vorwort der VWA formuliert wird. Es empfiehlt sich, für jedes
% andere Kapitel eine neue Datei anzulegen, um Umstellungen in der VWA (im Haupt-
% dokument "vwa.tex" im Hauptordner des Projekts) zu erleichtern.

Hier wird das Vorwort formuliert. Dieser Text sollte dabei gelöscht werden.


% Inhaltsverzeichnis an dieser Stelle erzeugen
\tableofcontents

% Hauptinhalt des Dokuments deklarieren
\mainmatter
% Kapitel hier einbinden, bspw:
% %
% 1_einleitung.tex
% Beispieldatei mitgeliefert mit der vwa.cls von Alexander Leithner
% Dieses Dokument fällt nicht unter die LPPL, unter der die vwa.cls steht!
% 

% Datei, die das Schreiben von Kapiteln für die VWA in LaTeX demonstrieren
% soll. Alle Kapitel müssen im Hauptdokument "vwa.tex" im Hauptordner des
% Projekts wie in Zeile 108 gezeigt, eingebunden werden!

\chapter{Einleitung}

Dies ist die Einleitung der Arbeit.

\section{Abschnitt}

Dies ist ein Abschnitt in der Einleitung der Arbeit.

\subsection{Unterabschnitt}

Dies ist ein Abschnitt eines Abschnittes in der Einleitung der Arbeit.
 

% (Prozentzeichen entfernen, um Befehl gültig zu machen)

% Anhang des Dokuments deklarieren
\appendix
% Anhang-Trennseite hier erzeugen:
\addpart*{Anhang}

% Literaturverzeichnis hier erzeugen:
\printbibliography

% Abbildungsverzeichnis hier erzeugen:
\listoffigures

% Selbstständigkeitserklärung hier erzeugen:
\selbststaendigkeitserklaerung

% Dokumentenende markieren, es darf kein Inhalt folgen!
\end{document}
